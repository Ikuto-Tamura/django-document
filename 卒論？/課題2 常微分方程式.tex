\documentclass{jarticle}
\usepackage{amssymb, amsmath}
\usepackage[dvipdfmx]{graphicx}
\title{課題2 常微分方程式}
\author{田村郁人 学生番号1029335165}
\date{\today}
\begin{document}
\maketitle

\section{課題3}
オイラー法、アダムス・バッシュフォース法(2次)、アダムス・バッシュフォース法(3次)、ルンゲ・クッタ法(2次)、ルンゲクッタ法(4次)の5つの方法により、誤差関数をプロットした。\\

プロットした点は、上から、オイラー法(紫)、アダムス・バッシュフォース法(2次)(黄緑)、ルンゲクッタ法(2次)(明るい黄色)、アダムス・バッシュフォース法(3次)(水色)、ルンゲクッタ法(4次)(濃い黄色)である。\\

プロットした直線は、上から$y=x,y=x^2,y=x^3,y=x^4$である。\\x軸とy軸両方を対数スケールでとっているため、図ではそれぞれ$y=x,y=2x, y=3x,y=4x$として表示されている。\\



このようにしてプロットした点は、x軸とy軸を対数で取ると下の図のように直線に回帰できるので、誤差関数は、べき関数であることが推測される。\\


\begin{center}

 \includegraphics[width=16cm]{kadai3tugikoso.png}

\end{center}

また、オイラー法によりプロットした点は、$y=x$とほぼ平行である。よって、べき関数の指数は1であると予測される。
\\
アダムス・バッシュフォース法(2次)によりプロットした点は、$y=x^2$とほぼ並行である。よって、べき関数の指数は2であると予測される。\\
アダムス・バッシュフォース法(3次)によりプロットした点は、$y=x^3$とほぼ並行である。よって、べき関数の指数は3であると予測される。\\
ルンゲ・クッタ法(2次)によりプロットした点は、$y=x^2$とほぼ並行である。よって、べき関数の指数は2であると予測される\\
ルンゲ・クッタ法(4次)によりプロットした点は、$y=x^4$とほぼ並行である。よって、べき関数の指数は4であると予測される\\


\section{課題4}

微分方程式を解くことにより、厳密解は、\\
$u(t)=\frac{9 e^{-10 t}+1}{10}$であることが求められる。

クランク・ニコルソン法について:

$u'=-\alpha u+\beta$について、差分を取って、$f_n = -\alpha u_n + \beta$とする。\\
$u_n = u_{n-1}+ \frac{\Delta t}{2}(f_{n-1}+f_n)$\\
$= u_{n-1}+\frac{\Delta t}{2}\lbrace	(-\alpha u_{n-1} + \beta)+(-\alpha u_n + \beta)\rbrace	$\\
$=(1-\frac{\Delta t }{2}) \alpha u_{n-1} - \frac{\Delta t}{2} \alpha u_n + \Delta t \beta$\\

$u_n$について解くと、

$u_n = \cfrac{1- \cfrac{\Delta t}{2} \alpha}{1+\cfrac{\Delta t}{2} \alpha} u_{n-1} + \cfrac{\Delta t \beta}{1+\cfrac{\Delta t}{2} \alpha}$\\

$\cfrac{1- \cfrac{\Delta t}{2} \alpha}{1+\cfrac{\Delta t}{2} \alpha}は、\Delta t> 0, \alpha > 0 より絶対値が1より小さいことが計算できる$

よって、クランク・ニコルソン法は安定である。\\

$さらに、u_0 = 1, \alpha= 10, \beta = 1として、$\\

$u' = - 10  u + 1$\\

$u_n = \frac{1-5 \Delta t}{1 + 5 \Delta t } u_{n-1 }+  \frac{\Delta}{1 + 5 \Delta t}$

$u_n=u(n\Delta t)であることに注意して、同時刻における厳密解の値と数値計算により値の誤差を取る。$\\

$\Delta t = 0.03125の場合、$

\includegraphics[width=16cm]{crank03125.png}

$\Delta t = 0.5の場合、$

\includegraphics[width=16cm]{crank05.png}

$\Delta t = 0.2の場合、$

\includegraphics[width=16cm]{crank02.png}

$よって、 \frac{1-5 \Delta t}{1 + 5 \Delta t }が正、負、0全ての場合において、誤差がt→∞において0に収束することが確認できた。$

ルンゲ・クッタ法(2次)について:

$u_n = (\frac{\alpha^2}{2} \Delta t^2 - \alpha \Delta t +1 ) u_{n-1} + \beta \Delta t - \frac{\alpha \beta}{2}\Delta t ^2$と書ける。\\

よって、$ -1 < \frac{\alpha^2}{2} \Delta t^2 - \alpha \Delta t +1 < 1となる\Delta tの範囲を考える。$

$\frac{\alpha^2}{2} \Delta t^2 - \alpha \Delta t +1 < 1を解いて、0 < \Delta t < \frac{2}{\alpha}$\\

$\frac{\alpha^2}{2} \Delta t^2 - \alpha \Delta t +1 > -1は、全ての実数値 \Delta tにおいて成り立つことが簡単に確かめらる。$\\

$従って、0 <\Delta t < \frac{2}{\alpha} のとき、このアルゴリズムは安定である。 \\ 
\Delta t \geq \frac{2}{\alpha}のとき、このアルゴリズムは不安定である。 $ \\

$さらに、u_0 = 1, \alpha= 10, \beta = 1として、$\\

$u_n = (50 \Delta t ^2 - 10 \Delta t  + 1) u_{n-1}+ \Delta t  - 5 \Delta t ^2$

$また、このアルゴリズムが安定となるのは、 0 < \Delta t < 0.2 のときである$

$u_n=u(n\Delta t)であることに注意して、同時刻における厳密解の値と数値計算により値の誤差を取る。$\\

$また、(50 \Delta t ^2 - 10 \Delta t  + 1) は常に正であることが簡単に確かめられる。$\\

$ \Delta t = 0.03125 のとき、$

\includegraphics[width=16cm]{runrun003125.png}

$0.03125 <  0.2 より誤差は0に収束はずだが、実際そうなっていることが確認できた。$

$ \Delta t = 0.3 のとき、$

\includegraphics[width=16cm]{runrun03.png}

$0.3 >  0.2 より誤差は0に収束しないはずだが、、実際発散していることが確認できた。$

$ \Delta t = 0.2 のとき、$

\includegraphics[width=16cm]{runrun02.png}

$0.2 =  0.2 より誤差は0に収束しないはずだが、実際0.9に収束していることが確認できた。$

\section{課題5}
$u' = -2u + 1の厳密解は、\\
\frac{du}{dt} = -2u + 1$\\
$\frac{2 du}{2u -1} = - 2 dt$\\
$log | 2u -1| = -2t + C_1  (C_1は任意定数)$\\
$2u - 1 = C e^{-2t}     (C = \pm e^{C_1})$\\
$u = \frac{C e^ {-2t}+1}{2}$\\
$t→∞において、e^{-2t}→0であるから、u→\frac{1}{2}である$\\
これはCの値によらないので、任意の初期値に対して、uが有限値に収束することが分かった。\\

$f(t_{n-1}, u_{n-1}) = -2 u_{n-1} + 1$\\
$u_n = u_{n-2} + 2 \Delta t ( -2 u_{n-1} + 1)$\\
$u_n = -4 \Delta t u_{n-1} + u_{n-2} + 2 \Delta t$\\
$u_n - \frac{1}{2} = -4 \Delta t (u_n - \frac{1}{2}) + (u_{n-2}- \frac{1}{2})$\\
$a_n = u_n - \frac{1}{2}とおいて、a_{n+2} = -4\Delta t a_{n+1} + a_n$

$b_n = \begin{pmatrix}
a_n \\
a_{n+1}\\
\end{pmatrix}
とおくと、$\\

$b_{n+1} = \begin{pmatrix} 
0 & 1 \\
1 & -4 \Delta t\\
\end{pmatrix} b_n$\\ 
$\begin{pmatrix} 
0 & 1 \\
1 & -4 \Delta t\\
\end{pmatrix} をAとおき、Aの固有値を\lambda とすると、\lambda= -2 \Delta t \pm \sqrt{4 \Delta t ^2 +1}$\\
$\lambda =-2 \Delta t + \sqrt{4 \Delta t ^2 +1}$ の固有ベクトルは、\\
$\begin{pmatrix} 
1 \\
-2 \Delta t + \sqrt{4 \Delta t^2 +1}\\
\end{pmatrix} $\\
$\lambda= -2 \Delta t - \sqrt{4 \Delta t ^2 +1}$の固有ベクトルは、\\
$\begin{pmatrix} 
1 \\
-2 \Delta t - \sqrt{4 \Delta t^2 +1}\\
\end{pmatrix} $\\
$この二つの固有ベクトルを並べ、行列PをP = \begin{pmatrix}
1 & 1\\
-2 \Delta t + \sqrt{4 \Delta t^2 +1}& -2 \Delta t - \sqrt{4 \Delta t^2 +1}\\
\end{pmatrix}のようにすると、P^{-1}A Pは対角行列となる$\\
$P^{-1}A P = \begin{pmatrix} 
-2 \Delta t + \sqrt{4 \Delta t ^2 +1} & 0 \\
0 &-2 \Delta t - \sqrt{4 \Delta t^2 +1}\\

\end{pmatrix} $\\
$ここで、固有値\lambda= -2 \Delta t - \sqrt{4 \Delta t ^2 +1}の絶対値、2\Delta t + \sqrt{4 \Delta t ^2 +1}は、\Delta t > 0より任意の\Delta tについて1より大きいことがわかる。$\\
$a_nが発散してしまうので、a_nに\frac{1}{2}を足したu_nも発散してしまう。$
よって、このアルゴリズムは不安定であることが示された。

$実際、u(0)=1として、数値計算をしてみたが、発散したことが確認された。たとえば、以下は、\Delta t = 0.3のときの様子である。\\$

\includegraphics[width=16cm]{kadai5.png}

\section{課題7}

1.
x(t)を図示すると、以下のようになった。\\

\includegraphics[width=16cm]{lorentzx.png}

また、軌道は以下のようになった。\\

\includegraphics[width=16cm]{xyzlor.png}

\section{課題8}

1.

$sin(x_i -x_j)= sin x_i cos x_j - sin x_j cos x_iである。$\\
$\frac{dx_i}{dt} = w_i - \frac{K}{N}\sum_{j = 1}^N (sin x_i cos x_j - sin x_j cos x_i)$\\
$\frac{dx_i}{dt} = w_i - sin x_i \frac{K}{N} \sum_{j = 1}^N  cos x_j + cos x_i \frac{K}{N} \sum_{j=1}^N  sin x_j $\\
$したがって、まず初めに、\sum_{j = 1}^N  cos x_jと \sum_{j=1}^N  sin x_j を計算する。$
その後、各iについて計算を行う。よって、計算はN+2回、O(N)で計算可能なことがわかる。



\end{document}




